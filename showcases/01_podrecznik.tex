\documentclass[11pt,a4paper]{article}
\usepackage[T1]{fontenc}
\usepackage[utf8]{inputenc}
\usepackage[polish]{babel}
% microtype removed
\usepackage{amsmath,amssymb,amsthm}
\usepackage[margin=2.5cm]{geometry}
\usepackage{tikz}
\usepackage{pgfplots}
\pgfplotsset{compat=1.18}
\usepackage{fancyhdr}
\usepackage{xcolor}
\usepackage{tcolorbox}
\tcbuselibrary{theorems,breakable}

\definecolor{accent}{RGB}{139,26,26}
\definecolor{thmbg}{RGB}{250,248,244}
\definecolor{defbg}{RGB}{240,245,250}

% Theorem styles
\newtcbtheorem[number within=section]{twierdzenie}{Twierdzenie}{
  colback=thmbg, colframe=accent, colbacktitle=accent,
  fonttitle=\bfseries, arc=1pt, boxrule=0.5pt,
  separator sign={\ ---},
  description delimiters parenthesis
}{tw}

\newtcbtheorem[number within=section]{definicja}{Definicja}{
  colback=defbg, colframe=accent!40!black, colbacktitle=accent!40!black,
  fonttitle=\bfseries, arc=1pt, boxrule=0.5pt,
  separator sign={\ ---},
  description delimiters parenthesis
}{def}

% Page style
\pagestyle{fancy}
\fancyhf{}
\fancyhead[L]{\small\textsc{Rozdział 3. Rachunek różniczkowy}}
\fancyhead[R]{\small\textsc{Analiza Matematyczna}}
\fancyfoot[C]{\thepage}
\fancyfoot[R]{\footnotesize\texttt{sklad-tekstu.pl}}
\renewcommand{\headrulewidth}{0.4pt}
\renewcommand{\footrulewidth}{0.2pt}

\setcounter{section}{2}
\setcounter{page}{87}

\begin{document}

\section{Rachunek różniczkowy funkcji jednej zmiennej}

\begin{definicja}{Pochodna funkcji}{pochodna}
Niech $f\colon (a,b) \to \mathbb{R}$ i~niech $x_0 \in (a,b)$. \textbf{Pochodną} funkcji~$f$ w~punkcie~$x_0$ nazywamy granicę
\begin{equation}\label{eq:pochodna}
  f'(x_0) = \lim_{h \to 0} \frac{f(x_0 + h) - f(x_0)}{h},
\end{equation}
o~ile granica ta istnieje i~jest skończona.
\end{definicja}

\begin{twierdzenie}{Reguła de l'H\^{o}pitala}{lhopital}
Jeśli $f$ i~$g$ są różniczkowalne na przedziale $(a,b)$, $g'(x) \neq 0$ dla $x \in (a,b)$ oraz
\[
  \lim_{x \to a^+} f(x) = \lim_{x \to a^+} g(x) = 0,
\]
to zachodzi
\begin{equation}\label{eq:lhopital}
  \lim_{x \to a^+} \frac{f(x)}{g(x)} = \lim_{x \to a^+} \frac{f'(x)}{g'(x)},
\end{equation}
o~ile granica po prawej stronie istnieje (właściwa lub niewłaściwa).
\end{twierdzenie}

\begin{proof}
Rozszerzamy $f$ i~$g$ na $[a,b)$, kładąc $f(a) = g(a) = 0$. Dla $x \in (a,b)$ stosujemy twierdzenie Cauchy'ego o~wartości średniej do $[a,x]$: istnieje $c \in (a,x)$ takie, że
\[
  \frac{f(x)}{g(x)} = \frac{f(x) - f(a)}{g(x) - g(a)} = \frac{f'(c)}{g'(c)}.
\]
Gdy $x \to a^+$, mamy $c \to a^+$ (gdyż $a < c < x$), zatem
$\displaystyle\lim_{x \to a^+} \frac{f(x)}{g(x)} = \lim_{c \to a^+} \frac{f'(c)}{g'(c)}$.
\end{proof}

\vspace{0.3cm}

\noindent\textbf{Przykład 3.1.} Oblicz $\displaystyle\lim_{x \to 0} \frac{\sin x - x}{x^3}$.

\medskip\noindent\textit{Rozwiązanie.} Trzykrotnie stosujemy regułę de l'H\^{o}pitala:
\[
  \lim_{x \to 0} \frac{\sin x - x}{x^3} 
  = \lim_{x \to 0} \frac{\cos x - 1}{3x^2}
  = \lim_{x \to 0} \frac{-\sin x}{6x}
  = \lim_{x \to 0} \frac{-\cos x}{6}
  = -\frac{1}{6}.
\]

\vspace{0.2cm}

\begin{minipage}[t]{0.48\textwidth}
\noindent\textbf{Interpretacja geometryczna pochodnej.}
Pochodna $f'(x_0)$ wyraża tangens kąta nachylenia stycznej do wykresu funkcji $f$ w~punkcie $(x_0, f(x_0))$. Na rysunku obok prosta styczna (czerwona) aproksymuje krzywą w~otoczeniu punktu $x_0$.
\end{minipage}
\hfill
\begin{minipage}[t]{0.48\textwidth}
\centering
\begin{tikzpicture}[scale=0.85]
\begin{axis}[
  axis lines=center,
  xlabel={$x$}, ylabel={$y$},
  xmin=-0.5, xmax=4, ymin=-0.5, ymax=5,
  xtick={1.5}, xticklabels={$x_0$},
  ytick=\empty,
  width=6.5cm, height=5.5cm,
  every axis x label/.style={at={(ticklabel* cs:1)}, anchor=west},
  every axis y label/.style={at={(ticklabel* cs:1)}, anchor=south},
]
\addplot[domain=0:3.8, smooth, thick, black] {0.3*x^2 + 0.5};
\addplot[domain=0.3:3, accent, thick] {0.9*x - 0.28};
\addplot[only marks, mark=*, mark size=2pt, accent] coordinates {(1.5, 1.175)};
\node[above right, font=\footnotesize] at (axis cs:1.5, 1.175) {$(x_0, f(x_0))$};
\node[right, font=\footnotesize, accent] at (axis cs:2.8, 2.24) {styczna};
\end{axis}
\end{tikzpicture}
\end{minipage}

\end{document}
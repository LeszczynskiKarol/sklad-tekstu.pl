\documentclass[11pt,a4paper]{article}
\usepackage[T1]{fontenc}
\usepackage[utf8]{inputenc}
\usepackage[polish]{babel}
% microtype removed
\usepackage[margin=2.5cm]{geometry}
\usepackage{booktabs}
\usepackage{siunitx}
\usepackage{pgfplots}
\pgfplotsset{compat=1.18}
\usepackage{xcolor}
\usepackage{fancyhdr}
\usepackage{tabularx}
\usepackage{caption}
\usepackage{float}

\definecolor{accent}{RGB}{139,26,26}
\definecolor{blue1}{RGB}{44,82,130}
\definecolor{green1}{RGB}{56,118,29}
\definecolor{orange1}{RGB}{204,102,0}
\definecolor{gray80}{gray}{0.2}
\definecolor{gray50}{gray}{0.5}

\sisetup{
  output-decimal-marker={,},
  group-separator={\,},
  group-minimum-digits=4
}

\pagestyle{fancy}
\fancyhf{}
\fancyhead[L]{\small\textsc{Raport techniczny RT-2024/037}}
\fancyhead[R]{\small\textsc{Analiza wydajności systemów PV}}
\fancyfoot[L]{\footnotesize Wersja 2.1 --- Styczeń 2025}
\fancyfoot[C]{\thepage}
\fancyfoot[R]{\footnotesize\texttt{sklad-tekstu.pl}}
\renewcommand{\headrulewidth}{0.4pt}
\renewcommand{\footrulewidth}{0.2pt}

\setcounter{section}{3}
\setcounter{page}{24}
\setcounter{table}{5}
\setcounter{figure}{8}

\begin{document}

\section{Analiza porównawcza wydajności instalacji}

\subsection{Produkcja energii w~cyklu rocznym}

Na rysunku~\ref{fig:produkcja} przedstawiono miesięczną produkcję energii elektrycznej trzech analizowanych instalacji fotowoltaicznych w~roku 2024. Instalacja~A (monokrystaliczna, \SI{10}{kWp}) osiągnęła najwyższą roczną produkcję \SI{11248}{kWh}, co~odpowiada współczynnikowi wydajności $PR = \num{0.842}$. Instalacja~B (polikrystaliczna, \SI{10}{kWp}) wygenerowała \SI{10115}{kWh} ($PR = \num{0.757}$), natomiast instalacja~C (cienkowarstwowa CdTe, \SI{10}{kWp}) --- \SI{9876}{kWh} ($PR = \num{0.739}$).

\begin{figure}[H]
\centering
\begin{tikzpicture}
\begin{axis}[
  width=0.92\textwidth,
  height=6.2cm,
  ybar,
  bar width=6pt,
  enlarge x limits=0.06,
  ylabel={\footnotesize Produkcja [kWh]},
  xlabel={\footnotesize Miesiąc},
  symbolic x coords={Sty,Lut,Mar,Kwi,Maj,Cze,Lip,Sie,Wrz,Paź,Lis,Gru},
  xtick=data,
  x tick label style={font=\tiny, rotate=0},
  y tick label style={font=\tiny},
  ylabel style={font=\footnotesize},
  xlabel style={font=\footnotesize},
  ymin=0, ymax=1700,
  legend style={at={(0.5,-0.22)}, anchor=north, legend columns=3, font=\footnotesize, draw=none},
  every axis plot/.append style={fill opacity=0.85},
  grid=major,
  grid style={gray!20},
  major tick length=2pt,
]
\addplot[fill=accent, draw=accent!80!black] coordinates {
  (Sty,320) (Lut,450) (Mar,780) (Kwi,1100) (Maj,1350) (Cze,1480)
  (Lip,1520) (Sie,1380) (Wrz,1050) (Paź,680) (Lis,380) (Gru,258)
};
\addplot[fill=blue1, draw=blue1!80!black] coordinates {
  (Sty,280) (Lut,395) (Mar,690) (Kwi,985) (Maj,1210) (Cze,1340)
  (Lip,1365) (Sie,1240) (Wrz,940) (Paź,610) (Lis,340) (Gru,220)
};
\addplot[fill=green1, draw=green1!80!black] coordinates {
  (Sty,290) (Lut,385) (Mar,665) (Kwi,960) (Maj,1175) (Cze,1305)
  (Lip,1330) (Sie,1205) (Wrz,910) (Paź,595) (Lis,335) (Gru,221)
};
\legend{Instalacja A (mono-Si), Instalacja B (poli-Si), Instalacja C (CdTe)}
\end{axis}
\end{tikzpicture}
\caption{Miesięczna produkcja energii elektrycznej analizowanych instalacji PV w~2024~r.}
\label{fig:produkcja}
\end{figure}

\subsection{Parametry techniczne i~wskaźniki ekonomiczne}

Tabela~\ref{tab:parametry} zawiera zestawienie kluczowych parametrów technicznych i~ekonomicznych badanych instalacji. Wartość LCOE (Levelized Cost of~Energy) obliczono przy założeniu 25-letniego okresu eksploatacji i~stopy dyskontowej $r = \SI{5}{\%}$.

\begin{table}[H]
\centering
\caption{Porównanie parametrów technicznych i~ekonomicznych instalacji PV.}
\label{tab:parametry}
\footnotesize
\begin{tabular}{@{}l S[table-format=2.1] S[table-format=5.0] S[table-format=1.3] S[table-format=3.1] S[table-format=1.2]@{}}
\toprule
\textbf{Parametr} & {\textbf{Moc [kWp]}} & {\textbf{Produkcja [kWh/rok]}} & {\textbf{PR [--]}} & {\textbf{LCOE [zł/kWh]}} & {\textbf{IRR [\%]}} \\
\midrule
Instalacja A (mono-Si)  & 10.0 & 11248 & 0.842 & 0.31 & 12.4 \\
Instalacja B (poli-Si)  & 10.0 & 10115 & 0.757 & 0.35 & 10.8 \\
Instalacja C (CdTe)     & 10.0 &  9876 & 0.739 & 0.38 &  9.6 \\
\midrule
\textit{Średnia}        & 10.0 & 10413 & 0.779 & 0.35 & 10.9 \\
\bottomrule
\end{tabular}
\end{table}

\noindent Instalacja~A wyróżnia się najniższym kosztem jednostkowym produkcji energii (LCOE~$= \SI{0.31}{zł/kWh}$) oraz najwyższą wewnętrzną stopą zwrotu ($\text{IRR} = \SI{12.4}{\%}$), co potwierdza przewagę technologii monokrystalicznej w~warunkach klimatycznych centralnej Polski. Szczegółową analizę degradacji modułów w~funkcji czasu eksploatacji przedstawiono w~rozdziale~5.

\end{document}